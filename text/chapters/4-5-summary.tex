\section{Summary}

The library \code{lox} consists of more than the parts discussed in the previous sections.
A brief overview of \code{lox} is presented in this summary.

The project consists of three crates: \code{lox} (the main one), \code{lox-macros} and \code{lox-cli}.
The crate \code{lox-macros} is a library crate defining some procedural macros, which are reexported by \code{lox}.
It exists for a purely technical reason: currently, Rust requires procedural macros to be defined in their own crate.
This restriction is likely to be lifted in the future, meaning the procedural macro definitions will be moved to the crate \code{lox} and \code{lox-macros} will be removed.

Unlike the others, \code{lox-cli} is an application crate that is not part of the library \code{lox}, i.\,e. it is not included in external projects that use \code{lox}.
Instead, it exists for three different purposes: it serves as an example for how to use \code{lox}, it is used as a test for \code{lox}'s capabilities and it is supposed to become a handy helper application for everyone working with meshes.
Currently, it supports converting meshes between different files formats (the ones supported by \code{lox}) and displaying various information about a mesh file.
Both operations can be tweaked via various command line parameters.
More features will be added in the near future.

The following list shows \code{lox}'s modules, excluding very small ones.
Modules marked with~* are or will be optional modules that can be disabled via Cargo features.

\begin{itemize}
  % TODO: maybe add more algorithms if you manage to implement more
  \item \textbf{\codebox{algo}}*: will offer various different mesh algorithms.
  Currently very incomplete: only a couple of example algorithms are implemented, including the subdivision algorithm \enquote{sqrt3} \cite{kobbelt20003}, Dijkstra's algorithm \cite{dijkstra1959note} and a simple smoothing algorithm.
  \item \textbf{\codebox{cast}}: utilities for casting numerical types (see \autoref{chap:io}).
  \item \textbf{\codebox{ds}}: implementations of shared vertex mesh, half edge mesh and directed edge mesh (see \autoref{chap:mesh-traits}) plus various helper utilities.
  \item \textbf{\codebox{fat}}: fat meshes (see \autoref{chap:io}).
  \item \textbf{\codebox{handle}}: handle types for faces, edges and vertices, the \code{Handle} trait and some utility functions (cf. \autoref{chap:memory-layout}).
  \item \textbf{\codebox{io}}*: IO traits, implementation of file formats (currently \textsc{Ply} and \textsc{Stl}) and other IO related functionality.
  \item \textbf{\codebox{map}}: property map traits and implementations (cf. \autoref{chap:memory-layout}).
  \item \textbf{\codebox{prop}}: abstractions over everything that can be treated like a 3D position (\code{Pos3Like}), a 3D vector (\code{Vec3Like}) or a color (\code{ColorLike}).
  \item \textbf{\codebox{refs}}: defines types that store a handle and a reference to the mesh, therefore representing a reference to a specific element of a mesh.
  These types expose various convenience methods which are defined in terms of methods of the mesh.
  \item \textbf{\codebox{shape}}*: will offer different shape definitions that can be used as a \code{StreamSource} to easily crate meshes from abstract shape definitions.
  Currently, only UV-spheres, 2D-discs and tetrahedra are implemented.
  \item \textbf{\codebox{traits}}: defines all mesh traits (cf. \autoref{chap:mesh-traits}).
\end{itemize}

\chapter{Implementation of \enquote{lox}}

This chapter discusses the major design decisions and notable implementation details behind the library \code{lox}.

% TODO: one crate, many Cargo features
The library mainly consists of a single crate (one compilation unit) which is very common for small to medium-sized Rust libraries.
As \code{lox} contains a variety of features, it is likely that some users do not need all of them.
To avoid compiling parts of the library that are unused, \code{lox} uses \emph{Cargo features} to let the user disable those parts individually.
For example, if a user does not read from or write to files, the IO module can be disabled completely and thus will not be compiled.
% TODO: add some feature gates

% TODO: maybe put this in a box or make it stand out a bit?
As an aside:
the term \enquote{lox} is often used as an abbreviation for \emph{\textbf{l}iquid \textbf{ox}ygen}, a pale-blue liquid usually obtained by cooling elemental oxygen below its boiling point of 90.188\,K.
This substance is used in many areas, playing a particularly important role in the aerospace industry where it is commonly used as a cryogenic oxidizer in rockets.
Since rust (as a chemical compound\footnote{The programming language \emph{Rust} is not even named after the chemical compound, but after the \enquote{Rust} family of fungi. Source: \url{https://www.reddit.com/r/rust/comments/27jvdt/}}) is a product of an oxidation reaction and rockets are usually very fast, using \enquote{lox} as name for this library seemed fitting.
However, the name was mainly chosen because it is short and pronounceable.

% TODO: lox logo?

\vspace{1cm}

\section{Basic Memory Layout Considerations}

\subsubsection*{Element Reference}

In many situations, it is necessary to refer to a specific element in the mesh (for example, to delete a specific face or to query the neighbors of a specific vertex).
There are mainly two ways to realize an \emph{element reference}: pointers or IDs/indices.
As discussed in chapter~2, it is desirable to use growing arrays to store mesh data, making pointers impossible (or at least very difficult) to use as element references.
As a consequence, \code{lox} uses the index of an element in its array to refer to that element.
(Also refer to \cite{sieger2011design} for a discussion about array vs. linked lists and indices vs. pointers.)

Using simple integers as element references in the library's API is not a good idea, however, since it is very easy to use an integer referring to one element kind (e.\,g. a vertex) in a context where a reference to another element kind (e.\,g. a face) is expected.
To avoid this confusion, multiple distinct types are introduced: \code{VertexHandle}, \code{EdgeHandle} and \code{FaceHandle}.
All of those types consist of a single integer field (called the \emph{handle ID}), making them behave exactly like an integer on machine level -- thus having no runtime overhead.
If one is used in the place of another, a compiler \enquote{type mismatch} error is printed, pointing out the bug immediately.
Additionally, a \code{Handle} trait is defined to abstract over all handle types.

Another benefit of dedicated handle types is that the internal integer type can be changed easily (without changing it across the whole codebase).
The default integer type in \code{lox} is a \code{u32} which is sufficient for most meshes in practice.
However, large meshes with more than $2^{32}$ elements require a wider integer type.
With the already mentioned Cargo features, it is possible to configure \code{lox} to used \code{u64} integers in handles, which is sufficient for all meshes in the foreseeable future.
% TODO: actually implement this


\subsubsection*{Deleting Elements in an Array}

TODO

\subsubsection*{Property Maps}

As mentioned in chapter 2, it is often necessary to associated arbitrary data with specific mesh elements.
This data is called \enquote{property} in this thesis; vertex normals are a \enquote{vertex property}, for example.
While some properties live as long as the mesh itself (e.\,g. vertex positions or face colors),  others are temporary values that only live for the duration of one algorithm (e.\,g. a boolean \code{visited} flag).
These temporary properties should not be kept in memory longer than necessary, meaning that they cannot be stored with the elements directly.
Instead, \code{lox} uses external maps (mapping from element to property), called \emph{property maps}.

These maps can be implemented in a variety of ways, including:

\begin{itemize}
\item A growing array using the handle ID as index (this works because it is already used as an array index in the main mesh).
\item A hash map using the handle ID as key.
\item A list of \code{(handle, value)} pairs.
\end{itemize}

In most situations, the growing array is the best choice as accessing values has virtually no overhead (apart from effects due to caching).
However, in a few situations it is necessary to associated values with only a couple of elements.
Using a growing array for that purpose is not a good idea, since the length of the array does not depend on the number of values but on the highest handle ID.
So on average, a growing array would waste a lot of memory, making other implementations a viable choice.

Just like with the mesh data structure, it does not make sense to decide on one specific implementation of external maps.
Instead, all implementations are offered by \code{lox} and abstracted over via traits.
That way, the most fitting implementation can be chosen in every situation.
The corresponding types are currently called \code{VecMap} (growable array), \code{HashMap} and \code{TinyMap} (list of tuples).
The abstraction is realized with three traits to make it as flexible as possible:

\begin{description}
  \item [\codebox{PropMap}] This trait only requires a single method to be implemented.
  Its signature is \code{fn get(&self, handle: H) -> Option<Value>} where \code{H} is the handle type parameter (this is a simplified version of the actual signature, which is explained in chapter~6).
  Returning \code{None} means that no value is associated with the given handle/element.
  This trait represents a simple map from handle to optional value.
  \item [\codebox{PropStore}] Like \code{PropMap} but can additionally iterate over all handles and values.
  \item [\codebox{PropStoreMut}] Like \code{PropStore} but can additionally change, insert and remove values.
\end{description}

The three data structures mentioned above (growing array, hash map, list of pairs) implement all three traits.
The reason for using three traits instead of a single one was to allow for more exotic type to be used.
The following types are provided by \code{lox} and implement \code{PropMap}, but not the other traits:

\begin{itemize}
\item \code{EmptyMap}: contains no properties, always returns \code{None}.
\item \code{ConstMap}: stores a single value and returns that value for all elements/handles.
\item \code{FnMap}: stores a function object with the signature \code{(&self, H) -> Option<Value>} which is used to implement \code{PropMap}.
\end{itemize}

While these types might not seem particularly useful at first, they are immensely handy in some situations.
Of course, they can be replaced by simply preparing a growable array or hash map with the desired values.
This, however, costs memory and extra time.
Specifically, \code{FnMap} is very useful to generate properties without using memory.

As a motivating example, consider an algorithm that produces a float value for each face, thus returning \code{VecMap<FaceHandle, f32>}.
This float value needs to be visualized via colors, for example by simply mapping the float value to gray-scale colors.
With \code{FnMap} a face~$\rightarrow$~color mapping can be created easily without allocating any memory:

\begin{center}
  \begin{minipage}{.59\textwidth}
    \begin{rustcode}
      let face_values = algorithm(&mesh);
      let face_colors = FnMap(|fh| {
          face_values.get(fh).map(to_color)
      });
    \end{rustcode}
  \end{minipage}
\end{center}


\subsubsection*{\enquote{Array of Structs} vs. \enquote{Struct of Arrays}}

TODO

\chapter{Implementation of \enquote{lox}}

This chapter discusses the major design decisions and notable implementation details behind \code{lox}.
The last section contains a short overview over all features offered by the library. As already mentioned in the introduction, \code{lox} is licensed as MIT/Apache-2.0 and can be found here: \url{https://github.com/LukasKalbertodt/lox}.

% TODO: one crate, many Cargo features
The library consists of a single crate (one minor exception is explained later) which is very common for small to medium-sized Rust libraries.
As \code{lox} contains a variety of features, it is likely that some users do not need all of them.
To avoid compiling parts of the library that are unused, \code{lox} uses \emph{Cargo features} to let the user disable those parts individually.
For example, if a user does not read from or write to files, the IO module can be disabled completely and thus will not be compiled.
% TODO: add some feature gates

% TODO: maybe put this in a box or make it stand out a bit?
As an aside:
the term \enquote{lox} is often used as an abbreviation for \emph{\textbf{l}iquid \textbf{ox}ygen}, a pale-blue liquid usually obtained by cooling elemental oxygen below its boiling point of 90.188\,K.
This substance is used in many areas, playing a particularly important role in the aerospace industry where it is commonly used as a cryogenic oxidizer in rockets.
Since rust (as a chemical compound\footnote{The programming language \emph{Rust} is not even named after the chemical compound, but after the \enquote{Rust} family of fungi. Source: \url{https://www.reddit.com/r/rust/comments/27jvdt/}}) is a product of an oxidation reaction and rockets are usually fast, using \enquote{lox} as name for this library seemed fitting.
However, the name was mainly chosen because it is short and pronounceable.

% TODO: lox logo?

\vspace{1cm}

\section{Basic Memory Layout Considerations}

\subsubsection*{Element Reference}

In many situations, it is necessary to refer to a specific element in the mesh (for example, to delete a specific face or to query the neighbors of a specific vertex).
There are mainly two ways to realize an \emph{element reference}: pointers or IDs/indices.
As discussed in chapter~2, it is desirable to use growing arrays to store mesh data, making pointers impossible (or at least very difficult) to use as element references.
As a consequence, \code{lox} uses the index of an element in its array to refer to that element.
(Also refer to \cite{sieger2011design} for a discussion about array vs. linked lists and indices vs. pointers.)

Using simple integers as element references in the library's API is not a good idea, however, since it is very easy to use an integer referring to one element kind (e.\,g. a vertex) in a context where a reference to another element kind (e.\,g. a face) is expected.
To avoid this confusion, multiple distinct types are introduced: \code{VertexHandle}, \code{EdgeHandle} and \code{FaceHandle}.
All of those types consist of a single integer field (called the \emph{handle ID}), making them behave exactly like an integer on machine level -- thus having no runtime overhead.
If one is used in the place of another, a compiler \enquote{type mismatch} error is printed, pointing out the bug immediately.
Additionally, a \code{Handle} trait is defined to abstract over all handle types.

Another benefit of dedicated handle types is that the internal integer type can be changed easily (without changing it across the whole codebase).
The default integer type in \code{lox} is a \code{u32} which is sufficient for most meshes in practice.
However, large meshes with more than $2^{32}$ elements require a wider integer type.
With the already mentioned Cargo features, it is possible to configure \code{lox} to used \code{u64} integers in handles, which is sufficient for all meshes in the foreseeable future.
% TODO: actually implement this


\subsubsection*{Deleting Elements in a Growable Array}

A simple growing array cannot replace a linked list in all situations, since the latter allows to remove elements anywhere in the list in $\mathcal O(1)$ whereas the former only supports removing elements from the end.
Being able to remove arbitrary elements from a mesh required in many situations, so mesh data structures are expected to support it.

There are two possibilities to remove an element at any position inside an array: (a) shift all elements right of the deleted elements one to the left, and (b) store a \code{deleted} flag for each element which is set to \code{true} once an element is deleted.
Solution (a) has a complexity of $\mathcal O(n)$ for per remove-operation, making it too slow for most applications.
However, the real problem is that shifting elements invalidates their indices which are used as element reference.
As a consequence, solution (b) has to be used.

There are different ways to implement the \code{deleted} flag.
The easiest and most obvious solution in Rust would be to use \code{Vec<Option<T>>} instead of \code{Vec<T>}.
In practice, storing many \code{Option<T>} instances is only a good idea if \code{Option<T>} can use \emph{null value optimization}.
This is usually not the case in the context of mesh data structures, meaning that \code{Option<T>} is notably larger than \code{T}.
(Specifically, it is \code{align_of::<T>()} larger, which for structs storing \code{u32} indices is usually 4 bytes.)

% TODO: Maybe show different ways to store deleted flags

A popular alternative way to store \code{deleted} flags is using a bit-vector, a densely packed array of bits (storing 8 bits per byte).
It has the disadvantage of requiring a separate allocation which slightly increases the reallocation cost and could potentially double the number of cache misses per element access.
However, this is usually out-weight by the benefits:
the dense packing means that no memory is wasted and that many of these flags fit into the cache at once (for example, a standard-sized L1 cache with 256~KB of storage can fit over 2~million bits, a 64~byte cache line can fit 512 bits).

It is also possible to use special \emph{sentinel values} to represent a deleted element (this is the general case of null-value optimization, where the sentinel value is 0) or to interleave the bits with the actual values in blocks to avoid an additional allocation.
A proper analysis of these different storage methods would be appropriate, but was outside the scope of this thesis.
In a quick comparison of the \code{Vec<Option<T>>} and bit-vector versions, no large difference in performance was found.
As a consequence and for memory efficiency, the bit-vector implementation was chosen for \code{lox}.
This choice is not final, however, and the implementation could easily be replaced by another one.

The logic for growable arrays that allow deletions is implemented completely independently of \code{lox} in a library called \code{stable-vec}\footnote{\url{https://crates.io/crates/stable-vec}}.
This library is technically also not part of this thesis.






\subsubsection*{Property Maps}

As mentioned in chapter 2, it is often necessary to associated arbitrary data with specific mesh elements.
This data is called \enquote{property} in this thesis; vertex normals are a \enquote{vertex property}, for example.
While some properties live as long as the mesh itself (e.\,g. vertex positions or face colors),  others are temporary values that only live for the duration of one algorithm (e.\,g. a boolean \code{visited} flag).
These temporary properties should not be kept in memory longer than necessary, meaning that they cannot be stored with the elements directly.
Instead, \code{lox} uses external maps (mapping from element to property), called \emph{property maps}.

These maps can be implemented in a variety of ways, including:

\begin{itemize}
\item A growing array using the handle ID as index (this works because it is already used as an array index in the main mesh).
\item A hash map using the handle ID as key.
\item A list of \code{(handle, value)} pairs.
\end{itemize}

In most situations, the growing array is the best choice as accessing values has virtually no overhead (apart from effects due to caching).
However, in a few situations it is necessary to associated values with only a couple of elements.
Using a growing array for that purpose is not a good idea, since the length of the array does not depend on the number of values but on the highest handle ID.
So on average, a growing array would waste a lot of memory, making other implementations a viable choice.

Just like with the mesh data structure, it does not make sense to decide on one specific implementation of external maps.
Instead, all implementations are offered by \code{lox} and abstracted over via traits.
That way, the most fitting implementation can be chosen in every situation.
The corresponding types are currently called \code{VecMap} (growable array), \code{HashMap} and \code{TinyMap} (list of tuples).
The abstraction is realized with three traits to make it as flexible as possible:

\begin{description}
  \item [\codebox{PropMap}] This trait only requires a single method to be implemented.
  Its signature is \code{fn get(&self, handle: H) -> Option<Value>} where \code{H} is the handle type parameter (this is a simplified version of the actual signature, which is explained in chapter~6).
  Returning \code{None} means that no value is associated with the given handle/element.
  This trait represents a simple map from handle to optional value.
  \item [\codebox{PropStore}] Like \code{PropMap} but can additionally iterate over all handles and values.
  \item [\codebox{PropStoreMut}] Like \code{PropStore} but can additionally change, insert and remove values.
\end{description}

The three data structures mentioned above (growing array, hash map, list of pairs) implement all three traits.
The reason for using three traits instead of a single one was to allow for more exotic type to be used.
The following types are provided by \code{lox} and implement \code{PropMap}, but not the other traits:

\begin{itemize}
\item \code{EmptyMap}: contains no properties, always returns \code{None}.
\item \code{ConstMap}: stores a single value and returns that value for all elements/handles.
\item \code{FnMap}: stores a function object with the signature \code{(&self, H) -> Option<Value>} which is used to implement \code{PropMap}.
\end{itemize}

While these types might not seem particularly useful at first, they are immensely handy in some situations.
Of course, they can be replaced by simply preparing a growable array or hash map with the desired values.
This, however, costs memory and extra time.
Specifically, \code{FnMap} is very useful to generate properties without using memory.

As a motivating example, consider an algorithm that produces a float value for each face, thus returning \code{VecMap<FaceHandle, f32>}.
This float value needs to be visualized via colors, for example by simply mapping the float value to gray-scale colors.
With \code{FnMap} a face~$\rightarrow$~color mapping can be created easily without allocating any memory:

\begin{center}
  \begin{minipage}{.59\textwidth}
    \begin{rustcode}
      let face_values = algorithm(&mesh);
      let face_colors = FnMap(|fh| {
          face_values.get(fh).map(to_color)
      });
    \end{rustcode}
  \end{minipage}
\end{center}


\subsubsection*{\enquote{Array of Structs} vs. \enquote{Struct of Arrays}}

Having property maps to store mesh properties externally brings up an interesting consideration: is it faster to use one property map per property kind or to use as few different maps as possible (e.\,g. by clustering different property kinds).
This problem is very similar to the \enquote{Array of Structs} (\emph{AoS}) vs. \enquote{Struct of Arrays} (\emph{SoA}) choice.
The two different layout strategies are visualized in figure X (TODO).

Both layouts use almost the same amount of memory and do not differ in capabilities.
The only difference is in how they interact with CPU caches.
To find an answer to what strategy should be preferred, a number of different benchmarks were conducted.


In these benchmarks, many items with a few random \code{u32} fields were stored in either one of the two memory layouts.
The time required to sum up all or some of these values was measured to get an idea of the respective speeds.
The different benchmarks attempted to cover the following parameter space:

\begin{itemize}
  \item \textbf{Memory layout}: AOS or SOA.
  \item \textbf{Traversal kind}: random or sequential.
  In the sequential benchmarks, a loop simply iterates sequentially over all indices starting from 0.
  The random benchmarks used the first field's \code{u32} value as next index to jump to.
  This was done to not include the generation of random numbers in the measured time.
  Notably, these \code{u32} values were already generated to be valid indices (i.\,e. smaller than the total number of items) and thus, the timed code does not need to perform any additional work on these values.
  \item \textbf{Total number of items}: values 10,000, 100,000, 1,000,000 and 10,000,000 were tested.
  \item \textbf{Number of [used] fields per item}: 1:1, 2:1, 2:2, 4:1, 4:2, 4:4, 6:1, 6:3 and 6:6 were tested.
  The syntax X:Y means that each item has X many fields and Y many fields were included in the sum, the other fields were ignored.
\end{itemize}

The benchmark code and more details on the benchmark can be found in the repository of this thesis in the folder \code{benchmarks/memory-layout/}.
The benchmark uses the framework \emph{Criterion} \cite{criterion} which features a sophisticated statistical analysis of the raw measurements.
The actual measurement was performed on a completely idle \textsf{Thinkpad T-460p} with an \textsf{Intel i7-6700\,HQ} running \textsf{Ubuntu 18.04} and ran for approximately three hours.

%TODO: add and discuss results

\section{Data Structures and Mesh Traits}


\section{Input/Output Module}


\section{Optimizations}


\section{Overview of all Library Features}

\chapter{Background: Geometry Processing}

The field of \emph{geometry processing} is concerned with acquiring, manipulating and analysing 3D surfaces.
Applications include simulations, modeling in industrial context, biomedical processing, and real-time 3D graphics (mostly, but not exclusively, for entertainment purposes).

A 3D surface can be acquired through a variety of different means.
In many areas, humans manually model 3D objects with a modeling software (like Blender\footnote{Blender website: \url{https://www.blender.org/}}), while in many other cases, 3D surfaces are algorithmically created from mathematical definitions or 3D scans (like MRI\footnote{Magnetic resonance imaging} data or laser scans).
It is then often necessary to manipulate the surface in order to improve certain characteristics of it.
For example, laser scan data is often fairly noisy, so various post-processing algorithms are needed to remove artifacts caused by that noise.
For real-time 3D graphics, it is often beneficial to reduce the complexity of an object in order to improve rendering times.

To work with 3D surfaces, one first has to find a way to represent them.
Amongst the many very different kinds of representations (cf. \cite{botsch2010polygon}, chapter 1), the polygon mesh is one of the most widely used one.
A polygon mesh is a collection of vertices (points), edges (connecting two vertices) and faces (polygons bounded by edges).
It is important to note that polygon meshes can only exactly represent polyhedrons, so most of the time, they are just an approximation of the actual 3D surface.

\begin{figure}[b]
  \includesvg{cat-algo}
  \caption{A 3D model of a cat (center). The left and right meshes are the results of a \emph{simplification} and \emph{subdivision} operation, respectively. (Original model from \cite{catmodel})}
\end{figure}

There are two special forms of the polygon mesh of high practical relevance. A \emph{triangle mesh} is a polygon mesh where each face is triangular and a \emph{quad mesh}, all faces are quadrilaterals.
Restricting the faces like that often makes algorithms operating on that mesh a lot simpler (and thus potentially faster).
In particular, since triangles are the simplest polygon (a 2-simplex, in fact), many operations on triangle meshes are a lot faster than on a general polygon mesh.

% - image of mesh with vertices, edges, faces marked
% - explanation of requirements (manifold, orientable, ...)
% - image of non-manifold features


\section{Mesh Data Structures}
% - requirements of data structures (adj info)
% - what makes DS slow/fast? Caches, ...
% - list and short explanation of different data structures

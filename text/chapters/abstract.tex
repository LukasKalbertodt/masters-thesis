\chapter*{\hfill Abstract}

Polygon meshes can be represented in various ways using different data structures, each varying in its capabilities, memory consumption and performance characteristics.
However, instead of attempting to abstract over several data structures, most popular geometry processing libraries only implement one.
The open-source library \code{lox}, developed as part of this thesis, provides such an abstraction using the trait system of the Rust programming language.
Comparing \code{lox} to the existing libraries OpenMesh and PMP regarding factors such as execution speed shows that the trait-based abstraction did not incur any significant runtime overhead, making \code{lox}'s core design a viable option for geometry processing libraries.
While choosing Rust as the programming language mostly benefited the project, some of its limitations, like the lack of generic associated types, notably slowed down the development of \code{lox}.
